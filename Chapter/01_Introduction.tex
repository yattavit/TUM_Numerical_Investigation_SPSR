\chapter{Introduction}
\label{ch:01-Introduction}
Fixed bed reactors play an important role in the chemical industry owing to their excellent behaviour with regards to low pressure drop and constructional simplicity \cite{Warnecke2000}. They are applied in many unit operations such as cracking as well as catalytic and non-catalytic reactors. Gaining insight into fluid dynamic properties such as pressure drop and residence time, as well as species concentrations, is of great importance for optimal design of fixed bed reactors. Computational software packages such as OpenFOAM\textsuperscript{\textregistered} can be used to study the hydrodynamic characteristics by numerical simulation with the possibility to include equations for species transport. The preferred turbulence model used in this work is the $k-\omega$-SST model after Menter et al. \cite{Menter2001, Menter2003}. SPSRs are a special type of fixed bed reactor where the confining wall of the reactor is only slightly larger than the diameter of the catalytic particles \cite{Scott1974}, which recently experienced a renewed interest \cite{Klyushina2015,Moonen2017,Mueller2017}.

The validation of pressure drop-flow rate relations for fixed-bed reactors have been a research topic for decades and is still continuing to be researched today \cite{Erdim2015}. An exception to this is the single pellet string reactor among the myriad of fixed-bed reactor types being researched and used in the industry. In a SPSR, spherical pellets are stacked alternatively on top of one another within a confining cylinder having a diameter which is slightly larger than the pellet itself. Although previously introduced by Scott et al. in the 1970s \cite{Scott1974}, SPSRs did not gain much recognition until recent publications revealed a renewed interest in this reactor concept, particularly its use for catalyst testing \cite{Moonen2017,Mueller2017,Klyushina2015}.
%extensive research on the flow behavior of SPSRs was performed in an attempt to validate it's similarity to conventional packed-bed reactors. However 
%The need for performance evaluation of commercial catalysts at industrial process conditions albeit in a much smaller scale \cite{Heinemann1994} makes SPSRs the ideal candidate in terms of economical and safety reasons

In a previous publication by Fernengel et al. \cite{Fernengel2019}, the characterization of conversion and residence time behaviour in SPSRs was achieved using numerical simulation and a design criterion comparing the reactor's performance to ideal plug flow was derived. In a subsequent work by the same author, the pressure drop behaviour in laminar flow of presented SPSRs variations is elaborated and compared with common literature correlations.

Pressure drop is related to the energy dissipated through the reactor bed due to fluid flow \cite{Al-Dahhan1994}. It is a hydrodynamic parameter that is of great interest to engineers as it can be used to determine energy losses, compression equipment size, liquid holdup, gas-liquid interfacial region and mass transfer coefficients \cite{Al-Dahhan1995a,Wammes1991,Holub1992}, all of which are important for the design and operation of various reactors.

% information obtained from PRESSURE DROP AND LIQUID HOLDUP IN HIGH PRESSURE TRICKLE-BED REACTORS \cite{Al-Dahhan1994}
%Pressure drop represents the energy dissipated due to fluid flow through the reactor bed. It is important in determining the energy losses, the sizing of the compression equipment, liquid holdup, external contacting efficiency, gas-liquid interfacial area and mass transfer coefficients, etc. (Holub, 1990; Wammes et al., 1991a,b; Al-Dahhan, 1993).

In his thesis Ergun \cite{Ergun1952} analysed variables that led to pressure loss of flow across randomly packed beds. Amongst these variables investigated are fluid properties, velocity, particle characteristics as well as bed porosity. A correlation, coined as the Ergun equation, was then proposed for pressure drop in packed-beds and has since then remained as one of the most popular correlation for pressure drop.

The Ergun equation is divided into two terms affiliated with viscous and kinetic losses, each having a constant, which are commonly referred as the first and second Ergun coefficient. In laminar flow, the Ergun equation is reduced to the Blake-Kozeny equation due to predominant viscous losses \cite{Blake1922,Kozeny1927a}. In turbulent flow however, the first term becomes close to negligible compared to the second term which becomes dominant.
%\cite{DuPlessis1994}

It is important to note that the Ergun equation does not account for pressure losses owing to the confining wall. Pressure losses due to the confining wall become more prevalent and severe with the decrease in cylinder-to-particle diameter ratio $D/d$. As such, modifications to the original Ergun equation may have to be made to account for the low cylinder-to-particle diameter ratio which is a characteristic of SPSR. Following the approach by Fernengel et al. \cite{Fernengel2020} in which the wall was incorporated into the equivalent diameter with a weighting factor, similar modifications are also adapted in this work. Such modifications included the use of equivalent diameters as characteristic length \cite{Scott1974,Mehta1969}, adjustments to the Ergun coefficients \cite{Reichelt1972, Eisfeld2001, Fand1993}, and the usage of a new empirical correlation \cite{Guo2017} to resolve the limitation of applying the original Ergun equation in a SPSR.

In this work, a numerical investigation of pressure drop in SPSRs is performed using computational fluid dynamics (CFD) simulation with the open-source program OpenFOAM\textsuperscript{\textregistered}. The influence of fluid viscosity, flow velocity, diameter aspect ratio and scaling factor on the pressure drop of turbulent flow through SPSRs is evaluated and compared to results previously obtained in laminar flow \cite{Fernengel2020}.
A weighting factor is introduced to the equivalent diameter expression following Scott et al.'s approach \cite{Scott1974} in an attempt to scale the influence of the confining wall surface, thereby improving the agreement between simulation results and Ergun-based pressure drop correlation. The simulative pressure drop values were also cross-referenced and discussed with respect to literature correlations (cf. Erdim et al. \cite{Erdim2015}).

It should also be noted that the geometry of the confining cylinder, parameters to be varied during parameter study and inert fines in this work are largely similar to \cite{Fernengel2020}.
Ultimately, the goal is to confirm and extend the range and validity of adjustments to the pressure drop correlation proposed by \cite{Fernengel2020} in the turbulent region.

%As such, there is a need for further simulative and experimental research in SPSRs for the to prove it's validity, whether be it in the performance testing of industrial sized heterogenous catalysts. With further research, the end goal is to eventually develop a physical SPSR model with predictable parameters [pressure drop,  conversion rate, ...] based simulative and experimental results to be readily used in the industry owing to its safety and economical benefits as compared to conventional fixed bed reactors.
%- Enhance safety and reduce Manpower.
%introduction [2-3 pages. packed beds in chem eng, SPSR, what is being done in the paper, pressure drop in numerical simulation
%something along the line of the uses of SPSR, i.e. commonly used in a small lab scale size to test for mass diffusion analysis [able to maintain isothermal condition, HAS LINEAR ARRANGEMENT OF PARTICLES THEREFORE EQUAL CHEMICAL REACTION CONDITIONS ASSUMED FOR EACH PARTICLE] to be upscaled to larger plant sizes... etc.
%Advantages of Pellet string reactors/why they’re commonly used
%• Pellet string reactors have been preferentially used in lab-scale experiments over other forms of reactors to predict optimum reaction conditions for gas-liquid-solid-reactions owing to its ability to control flow conditions at the catalyst particle, ensuring constant conditions for all particles.
%• Another reason for its popularity amongst other types of reactor is owing to the easy scalability of pellet string reactors.
%• Allows for instant detection of changes with respect to the catalyst structure
%• Allows for easy control of temperature for maintenance of isothermal condition
%• 
%Significance of Pellet string reactors
%• Improved wall surface to volume ratio with respect to fixed bed reactors
%◦ Better heat control and maintenance of isothermal condition within the reactor
%• The effects of having an almost linear arrangement of particles within the tube are twofold. Firstly, this encourages a large interfacial area for efficient contact between both the particles and the gas and liquid phase. Ultimately, flow channels and wall effects commonly observed in larger fixed beds can be eliminated owing to this arrangement. This theoretically means that equal chemical reaction conditions can be assumed for each particle.

%Equation 3-98 gives the total energy loss in fixed beds as the sum of viscous energy loss (the first term on the right side of the equation) and the kinetic or turbulent energy loss (the second term on the right side of the equation). For gas systems, approximately 80 of the energy loss is dependent on turbulence and can be represented by the second term of Equation 3-98. In liquid systems, the viscous term is the major factor.
%blake correlation Therein, the pressure drop is split into two additive terms related to viscous and kinetic losses, each with an empirical constant, commonly referred to as first and second Ergun coefficient, respectively. For laminar flow, viscous losses are predominant, reducing the Ergun equation to the first term, i.e., the Blake‐Kozeny equation 8, 9. 
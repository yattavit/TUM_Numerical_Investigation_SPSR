\chapter{Conclusion}
The simulated pressure drop across SPSRs mostly follows the expected trends previously established by Fernengel et al. in laminar flow \cite{Fernengel2020} upon varying fluid viscosity, flow velocity, diameter aspect ratio and reactor scale. Exceptions to the expected trends occurred with the variation of superficial velocity and scaling factor, in which the expected behaviour of pressure drop obtained with an increase in scaling factor was completely opposite to what was previously obtained in laminar flow \cite{Fernengel2020}. 

Although the graph obtained with superficial velocity in this work projects an exponential curve with increasing pressure drop as opposed to an expected linear trend as previously observed in laminar flow \cite{Fernengel2020}, its behaviour is still substantiated with findings presented by M\"{u}ller et al. \cite{Mueller2012}, who had also obtained similar results with a SPSR in turbulent flow within a simulative environment.

When considering the effects of the confining wall in turbulent flow, Ergun-based correlations provides a means to predict pressure drop behaviour.

The choice of base case geometry this time however, does not correspond to an agreement of most literature pressure drop correlations, with many of them underpredicting the pressure drop by 50\% or greater. Scott et al.'s \cite{Scott1974} method and addition of a weighting factor to the surface of the confining wall in the equivalent diameter formulation was followed, shifting results that are closer to parity in conjunction with the Ergun equation. 

Adopting a straightforward factor ($f_{w,1}$) based solely on the diameter aspect ratio on the base case, and cases with different diameter aspect ratio resulted in a substantial improvement in accuracy. Using a more detailed mathematical formulation for the weighting factor as a function of the diameter aspect ratio comprising of five designed parameters ($f_{w,2}$) across cases with varying diameter aspect ratios yielded results with differing successes in the prediction of pressure drop in SPSRs. However, it was still concluded nonetheless in the limiting case of SPSRs with differing diameter aspect ratios that the use of either suggested weighting factors will inadvertently push pressure drop correlation values to simulation values.

Overall, this work has mostly achieved its goal in obtaining simulative pressure drop in turbulent flow of the previously investigated SPSRs variations proposed by Fernengel et al. \cite{Fernengel2020}. Further fine-tuning of the $fvSolution$ code \ref{src:fvSolution} used is necessary to handle more turbulent flows (corresponding to increased wall modified Reynolds value) as well as cases with a larger reactor scale.
%GAMGSolverError occurs during simulation with E2
The obtained results were compared and contrasted against a deluge of pressure drop correlations from literature, most of which underpredicting the simulative pressure drop values. Thus, there is a need for more specialised pressure drop correlations to be developed (ideally through experiments in an SPSR) should a more accurate prediction of pressure drop in SPSRs be sought after.

Having more in-depth knowledge about crucial plant design parameters such as pressure drop behaviour will lead to better physical models of SPSRs following plug-flow behaviour to be developed. This may in turn lead to SPSRs to be adopted in the industries as a safer, cheaper and less labour intensive alternative to conventional packed bed reactors for the performance testing of catalysts.
%Talk about validity of study, fvSolution, fvschemes and snappyhexmesh, y+ all valid cause follows the graph \ref{fig:ComparisonAcrossSPSRVariations}.
%Talk about how plotting fp/pressure drop against wall mod Re or Re will affect only values obtained, but not trends observed with regards which correlation is closer to simulative results.